
\section{Results}
\label{sec:results}

Present the results of your experiments. Simply presenting the data is
insufficient! You need to analyze your results. What did you discover?
What is interesting about your results? Were the results what you
expected? Use appropriate visualizations. Prefer graphs and charts to
tables as they are easier to read (though tables are often more
compact, and can be a better choice if you're squeezed for space).


\begin{figure}[htb]
  \centering % centers the entire table

  % The following line sets the parameters of the table: we'll have
  % three columns (one per 'c'), each
  % column will be centered (hence the 'c'; 'l' or 'r' will left or
  % right justify the column) and the columns
  % will have lines between them (that's the purpose of the |s between
  % the 'c's).
  \begin{tabular}{|c|c|c|c|} 
    \hline \hline % draws two horizontal lines at the top of the table
    \rule{0pt}{4ex} d & A*(h1) & A*(h2) & A*(h3) \\ % separate column
                                % contents using the &
    \hline % line after the column headers
   \rule{0pt}{4ex} $2$ & $6$ & $6$ & $5$\\
    $4$ & $11$ & $11$ & $11$\\
    $6$ & $23$ & $18$ & $18$\\
    $8$ & $44$ & $28$ & $30$\\
    $10$ & $92$ & $41$ & $50$\\
    $12$ & $ $ & $ $ & $ $\\
    $14$ & $ $ & $ $ & $ $\\
    $16$ & $ $ & $ $ & $ $\\
    $18$ & $ $ & $ $ & $ $\\
    $20$ & $ $ & $ $ & $ $\\
    $22$ & $ $ & $ $ & $ $\\
    $ $ & $ $ & $ $ & $ $\\
    \hline \hline
  \end{tabular}

  % As with figures, *every* table should have a descriptive caption
  % and a label for ease of reference.
  \caption{Table 1. A visual representation of the respective heuristic performances.}
  \label{tab:example}

\end{figure}

