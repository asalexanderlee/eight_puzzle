
\section{Experiments}
\label{sec:expts}


We set about to replicate the experimental results of Russell and
Norvig regarding the 8-puzzle in order to shed some light on the
efficacy of different heuristics. In particular, we wanted to know the
effect a heuristic would have on a search algorithm. Therefore, we
implemented the A* search algorithm using three various heuristics:

\begin{itemize}
  \item h1 - the Misplaced Tiles heuristic, which returns the number
    of tiles that are not in their goal states on the given board
  \item h2 - the Manhattan Distance, which is defined as the sum of
    the distances of each tile from its goal state
  \item h3 - counts the number of tiles that are not in the correct
    rows and adds it to the number of tiles that are not in their
    correct columns. This could be considered a more relaxed version
    of the Manhattan Distance heuristic. 
\end{itemize}

In order to determine the quality of each heuristic, we recorded the
depth of the search, as well as the number of nodes generated. We then
ran a limited A* search for 20,000 trials in order to represent all
possible search depths from 1 to 20, and we calculated the effective
branching factor for each heuristic based on these results (shown in
Table 1).  






